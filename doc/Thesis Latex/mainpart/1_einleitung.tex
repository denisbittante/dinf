\documentclass[main.tex]{subfiles}
\begin{document}

%%%
%%%  Einleitung 
%%%

\chapter{Einleitung}





Die vorliegende Arbeit beschäftigt sich mit der Evaluation von \gls{osre} die Frage nach der performantesten \gls{osre} ist von besonderem Interesse weil immer häufiger \acrfull{SOA} im Einsatz sind.  




Die Evaluation von OSRE ist von Interesse, da sich die reine Evaluation der Performance auf die Infrastruktur auswirkt, umso effizienter und schneller die Engines die Anforderungen erfüllen, desto schwächer bzw. günstiger kann die Infrastruktur ausfallen was auch bedeutet, dass diese sich wirtschaftlich gesehen schneller rentieren werden. Im aktuellen Markt zeigt sich der Trend zu Containers und PaaS, welche horizontal Skalierbar sind und keine Grenzen (aus Sicht der Hardware) gesetzt werden.  Performance- und Lasttests  können in diesem Anwendungsfall (generieren der PDF's) aufzeigen, wie sich ein solcher Services im Extremszenario auf einem Knoten verhalten würde. Diese Informationen können unter Umständen Ausfälle und Leistungsengpässe einzelner Container oder Knoten im Verbund vorhersagen. Damit lassen sich Infrastrukturen planen und administrieren um diese z.B. in Bezug auf die aktuellen User-Aktivität auslegen auch wenn diese auf einer PaaS oder Iaas betrieben werden sollen. 



\end{document}