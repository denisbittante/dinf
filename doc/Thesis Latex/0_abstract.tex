\documentclass[main.tex]{subfiles}
\begin{document}




\begin{abstract}
\selectlanguage{\english}


%Background, 
Die Enterprise Software ist im Wandel und Microservices werden immer häufiger eingesetzt. 
% Objectives 
\noindent
Diese Arbeit widmet sich der Erforschung der geeigneten Open Source Reporting Engine (OSRE) im Hinblick auf mögliche Ursachen für Peformance-Engpässe und deren Einsatz im Umfeld von Container und \acrfull{iaas}. 

%Methods, 
\noindent
Die OSRE wurden in Prototypen genutzt, um verschiedene PDFs zu  erstellen. Die Prototypen wurden als REST-Services auf IaaS deployed und gegen einen Lasttest laufen gelassen. Die Tests gaben Aufschluss über Latenzzeit, Memory-Verbrauch und  Verfügbarkeit des Service. 

%Results, and 
\noindent
Die Ergebnisse haben gezeigt das iText als \acrlong{osre} ein performantes Tool mit gutem API und guter Nutzung der Ressoucen ist. Reine Performance wird mit Apache PDFBox besser erreicht, da dies Tool auf Performance getrimmt wurde hierbei fehlen aber Mittel um schöne oder komplexe Layout zu generieren. Maximaler Durchsatz konnte mit Apache PDFBox bei 151 Anfragen pro Sekunden verzeichnet werden. JasperReport hat sehr gut abgeschnitten im Layout und Usabilty-Teil aber ist kein Kandidat für performante Umsetzung von Services. Über alle Test hinweg hat JasperReport die niedrigste Performance erreicht.

%Conclusions
\noindent
Die Experimente deuten darauf hin, dass die Auswahl eines geeigneten Frameworks eine wichtige Rolle spielt, wenn eine hohe Performance erreicht werden soll.

%The primary take-home message
%The additional findings of importance
%The perspective



\end{abstract}



\end{document}